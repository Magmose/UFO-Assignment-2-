\documentclass{article}
\usepackage[utf8]{inputenc}
\usepackage{hyperref}
\bibliographystyle{abbrv}
\hypersetup{
    colorlinks=true,
    linkcolor=blue,
    filecolor=magenta,      
    urlcolor=blue,
}
\urlstyle{same}
\title{Requirements for a Bachelor Thesis}
\author{
    Mathias B. Kristensens\\
    \href{mailto:cph-mk523@cphbusiness.dk}{cph-mk523@cphbusiness.dk}
    Magnus Albeck Klitmose\\
    \href{mailto:cph-mk523@cphbusiness.dk}{cph-mk525@cphbusiness.dk}
    }
\date{February 2021}
\begin{document}
   \maketitle
   \section{Task 1}
\subsection{Formalities}
Certain formalities surround the bachelor thesis, which are specified in the curriculum \cite{curriculum}, 
as well as a specification for the bachelor thesis from 2020 \cite{thesisspec}: 
\begin{itemize}
    \item The thesis is always written on the student's 3rd semester
    \item It can be done individually or in groups of 2-4 people
    \item It covers 15 ECTS points
    \item It has a maximum page count of  \(40 + 20 * number Of Students \)..
    \item It can be written in either English or Danish
  \end{itemize}

\subsection{The project}
In broad terms, the student(s) \begin{quotation}
    "...investigates a softwarerelated problem, propose a solution, and does some implementation to demonstrate the solution. 
    The student should strive to include elements from the courses passed during the programme (Large Systems Development, Databases, Testing, System Integration)." 
    \cite{thesisspec} 
\end{quotation}
A full recommendation of suggested components of the report is also found in section 3.1 of \cite{thesisspec}. 



\begin{thebibliography}{9}
    \bibitem{curriculum} 
    \href{https://www.cphbusiness.dk/media/1177/pba_soft_cba_studieordning.pdf}{Curriculum for Software Development} 

    \bibitem{thesisspec}
    \href{https://datsoftlyngby.github.io/soft2020fall/resources/bbe51cf2-bachelorProject.pdf}{Thesis specification from 2020} 
\end{thebibliography}
\end{document}